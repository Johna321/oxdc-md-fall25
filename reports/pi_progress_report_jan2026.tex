\documentclass[11pt,letterpaper]{article}

% Packages
\usepackage[margin=1in]{geometry}
\usepackage{graphicx}
\usepackage{booktabs}
\usepackage{amsmath}
\usepackage{siunitx}
\usepackage{hyperref}
\usepackage{xcolor}
\usepackage{enumitem}
\usepackage{float}
\usepackage{subcaption}
\usepackage{fancyhdr}

% Colors
\definecolor{ufblue}{RGB}{0,33,165}
\definecolor{uforange}{RGB}{250,70,22}

% Header/Footer
\pagestyle{fancy}
\fancyhf{}
\rhead{OxDC MD Progress Report}
\lhead{John Aitken}
\rfoot{Page \thepage}

% Custom commands
\newcommand{\angstrom}{\text{\AA}}

% Title
\title{\textbf{OxDC Molecular Dynamics Simulations:\\Progress Report for Winter Break 2025-2026}}
\author{John Aitken\\
Department of Chemistry, University of Florida\\
\texttt{john.aitken@ufl.edu}}
\date{January 13, 2026}

\begin{document}

\maketitle

\begin{abstract}
This report summarizes progress on the oxalate decarboxylase (OxDC) molecular dynamics simulation project during the winter break period. The primary accomplishment is the completion and rigorous analysis of a \textbf{474 ns} production trajectory for the BiOx+2 system (bidentate oxalate with Mn(II)), achieving 467 ns/day throughput on NVIDIA B200 GPUs. \textbf{Key finding:} The active site lid remains in the ``closed-backbone / Glu162-out'' state throughout the simulation (Glu162-Mn = 12.0 $\pm$ 0.7~\AA), matching the 5VG3 crystal structure. This is distinct from both the ``open-loop'' state (1J58, $\sim$15-16~\AA) and the catalytically active ``Glu162-in closed'' state (1UW8, $\sim$4.6-5.1~\AA). Bidentate oxalate binding likely sterically prevents Glu162 from adopting the catalytically active Glu162-in pose. Block averaging across 231 two-nanosecond windows confirms convergence ($<$3\% SEM for all key metrics). Additionally, equilibration of the 1Wat+3 system (Mn(III)) revealed significant numerical instability due to elevated force constants, providing insight into the challenges of classical MD for Mn(III) metalloenzymes.
\end{abstract}

\tableofcontents
\newpage

%======================================================================
\section{Executive Summary}
%======================================================================

\subsection{Major Accomplishments}

\begin{enumerate}[label=\arabic*.]
    \item \textbf{Completed 474 ns BiOx+2 production simulation}
    \begin{itemize}
        \item Stable trajectory with no force field issues
        \item 467 ns/day performance on HiPerGator B200 GPU nodes
        \item Full cpptraj analysis pipeline executed (46,383 frames)
    \end{itemize}

    \item \textbf{Comprehensive trajectory analysis}
    \begin{itemize}
        \item Structural stability (Active site RMSD = 1.77 $\pm$ 0.42~\AA)
        \item Mn1 coordination integrity (0 dissociation events)
        \item Oxalate binding mode (asymmetric bidentate, 92.6\%)
        \item \textbf{Lid dynamics: 5VG3-like (Glu162-out) state throughout 474 ns}
    \end{itemize}

    \item \textbf{Statistical validation}
    \begin{itemize}
        \item Block averaging confirms convergence
        \item Correlation analysis identifies coupled motions
        \item Unimodal Glu162-Mn distribution (not transitioning)
    \end{itemize}

    \item \textbf{Force field parameter analysis}
    \begin{itemize}
        \item MCPB.py parameters validated against production trajectory
        \item k $<$ 35 kcal/mol$\cdot$\AA$^2$ threshold confirmed for stability
        \item Seminario method produces appropriate force constants
    \end{itemize}
\end{enumerate}

\subsection{Key Scientific Finding}

\begin{center}
\fbox{\parbox{0.9\textwidth}{
\centering
\textbf{The active site lid remains in the 5VG3-like ``closed-backbone / Glu162-out'' state}\\
\vspace{0.5em}
Glu162-Mn = 12.0 $\pm$ 0.7~\AA~$|$~Consistent with 5VG3 ($\sim$10-12~\AA)~$|$~Not 1UW8 Glu162-in ($\sim$4.6~\AA)
}}
\end{center}

This finding has significant mechanistic implications:
\begin{itemize}
    \item Glu162 is the essential proton donor (E162A mutation eliminates activity)
    \item At 12.0~\AA, direct proton transfer to Mn1 is geometrically impossible
    \item The Glu162-out conformation may be stabilized by bidentate oxalate (steric clash with Glu162-in)
    \item Transition to Glu162-in may require oxalate rearrangement or other triggering event
\end{itemize}

%======================================================================
\section{BiOx+2 474 ns Production Analysis}
%======================================================================

\subsection{Simulation Parameters}

\begin{table}[H]
\centering
\begin{tabular}{ll}
\toprule
\textbf{Parameter} & \textbf{Value} \\
\midrule
System & BiOx+2 (bidentate oxalate, Mn(II)) \\
PDB origin & 5VG3 \\
Topology & 5vg3\_solv.prmtop \\
Total atoms & 63,287 \\
Water molecules & 19,079 (TIP3P) \\
Ions & 1 Cl$^-$ \\
Temperature & 300 K (Langevin) \\
Pressure & 1 atm (MC barostat) \\
Timestep & 2 fs \\
Production length & 474 ns (237,000,000 steps) \\
Saved frames & 46,383 (10 ps/frame) \\
GPU & NVIDIA B200 (hpg-b200 partition) \\
GPU performance & 467 ns/day \\
\bottomrule
\end{tabular}
\caption{Simulation parameters for BiOx+2 production run.}
\end{table}

\subsection{Structural Stability}

The backbone C$\alpha$ RMSD shows conformational sampling throughout the 474 ns trajectory, while the active site remains highly stable:

\begin{table}[H]
\centering
\begin{tabular}{lrr}
\toprule
\textbf{Metric} & \textbf{Backbone} & \textbf{Active Site} \\
\midrule
Mean RMSD & 4.70 $\pm$ 1.84~\AA & 1.77 $\pm$ 0.42~\AA \\
Range & 0.00 -- 7.30~\AA & --- \\
Drift & +0.012~\AA/ns & negligible \\
Early ($<$10\%) mean & 2.41~\AA & --- \\
Late ($>$90\%) mean & 6.82~\AA & --- \\
\bottomrule
\end{tabular}
\caption{RMSD statistics for 474 ns production trajectory.}
\end{table}

The backbone RMSD reflects conformational sampling of flexible N- and C-terminal regions, as confirmed by RMSF analysis showing terminal RMSF $>$5~\AA. Critically, the \textbf{active site RMSD remains low} (1.77~\AA), indicating that the catalytic center maintains structural integrity throughout the simulation. The radius of gyration (24.02 $\pm$ 0.35~\AA) confirms the protein remains compact without unfolding.

\begin{figure}[H]
\centering
\includegraphics[width=0.95\textwidth]{../systems/BiOx+2/analysis_results/figures/prod_rmsd.png}
\caption{C$\alpha$ RMSD time series and distribution for 474 ns BiOx+2 production. Left: Backbone (blue) and active site (red) RMSD with 5 ns running average. Right: RMSD distributions showing stable active site despite backbone sampling.}
\label{fig:rmsd}
\end{figure}

\subsection{Mn1 Coordination Integrity}

The MCPB.py-parameterized Mn1 coordination sphere remained intact throughout the simulation (Figure~\ref{fig:mn1coord}):

\begin{table}[H]
\centering
\begin{tabular}{lrrl}
\toprule
\textbf{Ligand} & \textbf{Mean (\AA)} & \textbf{Std (\AA)} & \textbf{Dissociation Events} \\
\midrule
His95-NE2 & 2.42 & 0.12 & 0 \\
His97-NE2 & 2.27 & 0.09 & 0 \\
His140-NE2 & 2.22 & 0.09 & 0 \\
Glu101-OE1 & 2.06 & 0.09 & 0 \\
\bottomrule
\end{tabular}
\caption{Mn1-ligand distances from 474 ns production. All values within expected MCPB.py $r_0$ ranges.}
\end{table}

This validates the Seminario method force constants for this system (mean k = 29.7 kcal/mol$\cdot$\AA$^2$).

\begin{figure}[H]
\centering
\includegraphics[width=0.95\textwidth]{../systems/BiOx+2/analysis_results/figures/prod_mn1_coordination.png}
\caption{Mn1-ligand distance time series for all four protein ligands. Green shading indicates expected $r_0$ ranges from MCPB.py. Red dashed line shows 3.0~\AA\ dissociation threshold---never crossed.}
\label{fig:mn1coord}
\end{figure}

\subsection{Oxalate Binding Mode}

The oxalate substrate maintains asymmetric bidentate ($\kappa$O,$\kappa$O') coordination throughout (Figure~\ref{fig:oxalate}):

\begin{table}[H]
\centering
\begin{tabular}{lrl}
\toprule
\textbf{Oxygen} & \textbf{Mean Distance (\AA)} & \textbf{Classification} \\
\midrule
OZ & 2.09 $\pm$ 0.07 & Coordinating (tight) \\
OX & 2.35 $\pm$ 0.11 & Coordinating (loose) \\
\bottomrule
\end{tabular}
\caption{Mn1-oxalate oxygen distances. Bidentate fraction = 92.6\%.}
\end{table}

The asymmetric binding is consistent with:
\begin{itemize}
    \item $^{13}$C-ENDOR spectroscopy (Zhu et al., 2024): Confirmed bidentate binding in OxDC
    \item DFT calculations: Bidentate is 4.7 kcal/mol more stable than monodentate
    \item Our previous force constant analysis: The loose OX provides a ``shock absorber'' effect
    \item \textbf{Steric constraint on Glu162:} Bidentate oxalate may sterically clash with the Glu162-in pose (Zhu et al. 2016)
\end{itemize}

\begin{figure}[H]
\centering
\includegraphics[width=0.95\textwidth]{../systems/BiOx+2/analysis_results/figures/prod_oxalate_binding.png}
\caption{Oxalate binding analysis. Left: Distance time series. Center: Distance distributions. Right: OZ vs OX 2D scatter colored by time, showing stable asymmetric bidentate region.}
\label{fig:oxalate}
\end{figure}

\subsection{Lid Dynamics: The Key Finding}

\textbf{The Glu162 sidechain remains in the 5VG3-like ``Glu162-out'' position throughout the entire 474 ns simulation.}

\begin{table}[H]
\centering
\begin{tabular}{lr}
\toprule
\textbf{Metric} & \textbf{Value} \\
\midrule
Glu162 CD - Mn1 distance & 12.04 $\pm$ 0.70~\AA \\
Glu162 OE1 - Mn1 distance & 12.03 $\pm$ 0.95~\AA \\
Glu162 OE2 - Mn1 distance & 13.05 $\pm$ 0.68~\AA \\
\midrule
\multicolumn{2}{l}{\textbf{Three-State Classification (Literature):}} \\
Glu162-in fraction ($<$6~\AA) & 0.0\% \\
Glu162-out fraction (8-14~\AA) & 100.0\% \\
Open-loop fraction ($>$14~\AA) & 0.0\% \\
\midrule
Transitions toward Glu162-in & 0 \\
Closest approach & 8.23~\AA~at 229 ns \\
Block SEM (231 blocks) & 0.035~\AA~(0.29\%) \\
\bottomrule
\end{tabular}
\caption{Lid dynamics analysis. Glu162 maintains the 5VG3-like Glu162-out position throughout 474 ns.}
\end{table}

\begin{figure}[H]
\centering
\includegraphics[width=0.95\textwidth]{../systems/BiOx+2/analysis_results/figures/prod_lid_dynamics.png}
\caption{Lid dynamics analysis. Top left: Glu162-Mn1 distance time series showing persistent Glu162-out state. Top right: Distance histogram with three-state thresholds. Bottom left: OE1/OE2 comparison. Bottom right: Block averages confirming consistency across 2 ns windows.}
\label{fig:lid}
\end{figure}

\subsubsection{Comparison to Crystal Structures (Three-State Model)}

Literature establishes \textbf{three distinct lid states}:

\begin{table}[H]
\centering
\begin{tabular}{llrl}
\toprule
\textbf{State} & \textbf{PDB} & \textbf{Glu162-Mn (\AA)} & \textbf{Description} \\
\midrule
Open-loop & 1J58 & $\sim$15-16 & SENS loop swung away \\
Closed, Glu162-in & 1UW8 & $\sim$4.6-5.1 & Glu162 H-bonds Mn-water \\
Closed-backbone, Glu162-out & 5VG3 & $\sim$10-12 & Loop closed, sidechain out \\
\textbf{Our simulation (474 ns)} & \textbf{BiOx+2} & \textbf{12.0} & \textbf{5VG3-like} \\
\bottomrule
\end{tabular}
\caption{Glu162-Mn distance comparison with crystallographic data.}
\end{table}

\textbf{Key interpretation:} Our simulation maintains the 5VG3-like ``closed-backbone / Glu162-out'' state. This is:
\begin{itemize}
    \item \textbf{Not} the open-loop state (1J58, $\sim$15-16~\AA) where the SENS loop is swung away
    \item \textbf{Not} the catalytically active Glu162-in state (1UW8, $\sim$4.6-5.1~\AA)
    \item Consistent with starting from 5VG3 and maintaining that conformation
    \item Likely stabilized by bidentate oxalate sterically preventing Glu162-in
\end{itemize}

\textbf{Note:} Previous interpretations incorrectly classified $\sim$11.5~\AA\ as ``open.'' The three-state model clarifies that this is a distinct ``closed-backbone / Glu162-out'' state.

\subsubsection{Mechanistic Implications}

Glu162 is essential for catalysis---the E162A mutation eliminates decarboxylase activity (Saylor et al., 2008). Glu162 serves as the proton donor in the proposed PCET mechanism, requiring close proximity to the Mn-bound water ($\sim$2.7-2.8~\AA\ contact in 1UW8).

At 12.0~\AA, direct proton transfer is impossible. This suggests:

\begin{enumerate}
    \item \textbf{Bidentate oxalate sterically blocks Glu162-in} -- Consistent with Zhu et al. 2016 structural analysis showing the 1UW8 Glu162 pose clashes with substrate
    \item \textbf{Glu162-in may require substrate rearrangement} -- Monodentate binding could create space for Glu162 to adopt the catalytic pose
    \item \textbf{Backbone analysis needed} -- RMSD to 1J58 vs 1UW8 reference structures would confirm loop backbone state
\end{enumerate}

\subsection{Flexibility Analysis}

Per-residue RMSF analysis reveals that the lid region (160-166) is \textbf{less flexible than average}:

\begin{table}[H]
\centering
\begin{tabular}{lr}
\toprule
\textbf{Region} & \textbf{Mean RMSF (\AA)} \\
\midrule
Global average & 1.03 \\
\textbf{Lid (160-166)} & \textbf{0.71} \\
Active site (His95, 97, 140, Glu101) & 0.52-0.59 \\
\bottomrule
\end{tabular}
\caption{Regional RMSF comparison.}
\end{table}

This indicates that the Glu162-out conformation is \textbf{stabilized}, not fluctuating. The 5VG3-like state appears to be a genuine energy minimum when bidentate oxalate is bound.

\begin{figure}[H]
\centering
\includegraphics[width=0.95\textwidth]{../systems/BiOx+2/analysis_results/figures/prod_rmsf.png}
\caption{Per-residue RMSF analysis. Top: Full protein profile with lid region highlighted in red. Bottom: Lid residue RMSF bar chart.}
\label{fig:rmsf}
\end{figure}

\subsection{Convergence Assessment}

Block averaging (231 blocks of 2 ns each) confirms excellent simulation convergence:

\begin{table}[H]
\centering
\begin{tabular}{lrrr}
\toprule
\textbf{Metric} & \textbf{Mean} & \textbf{SEM} & \textbf{SEM \%} \\
\midrule
Backbone RMSD (\AA) & 4.70 & 0.120 & 2.56\% \\
Active Site RMSD (\AA) & 1.77 & 0.024 & 1.35\% \\
Radius of Gyration (\AA) & 24.02 & 0.022 & 0.09\% \\
Glu162-Mn (\AA) & 12.04 & 0.035 & 0.29\% \\
Mn1-His95 (\AA) & 2.42 & 0.002 & 0.07\% \\
Mn1-Glu101 (\AA) & 2.06 & 0.001 & 0.06\% \\
\bottomrule
\end{tabular}
\caption{Block averages for key metrics across 231 two-nanosecond windows. All SEM values $<$3\% of mean, indicating excellent convergence.}
\end{table}

\subsection{Correlation Analysis}

\begin{table}[H]
\centering
\begin{tabular}{lrl}
\toprule
\textbf{Correlation} & \textbf{r} & \textbf{Interpretation} \\
\midrule
RMSD vs Glu162-Mn & +0.08 & Weak -- lid independent of global motion \\
RMSD vs Lid RMSD & +0.08 & Weak -- lid fluctuations independent \\
Lid RMSD vs Glu162-Mn & +0.39 & Moderate -- lid motion coupled to position \\
\bottomrule
\end{tabular}
\caption{Pearson correlation coefficients between key metrics.}
\end{table}

\begin{figure}[H]
\centering
\includegraphics[width=0.95\textwidth]{../systems/BiOx+2/analysis_results/figures/prod_correlations.png}
\caption{Correlation scatter plots for key structural metrics.}
\label{fig:corr}
\end{figure}

%======================================================================
\section{1Wat+3 Equilibration: Mn(III) Challenges}
%======================================================================

The 1Wat+3 system (water-coordinated Mn(III)) was equilibrated to investigate oxidation state effects on lid dynamics. However, significant numerical instability was observed, providing insight into the limitations of classical MD for Mn(III) metalloenzymes.

\subsection{Equilibration Progress}

The system progressed through heating and equilibration stages but accumulated substantial velocity limit (vlimit) warnings:

\begin{table}[H]
\centering
\begin{tabular}{lrr}
\toprule
\textbf{Stage} & \textbf{vlimit Warnings} & \textbf{Status} \\
\midrule
heat.cpu & 2,467 & Completed \\
eq1.cpu & 6,174 & Completed \\
eq1a.cpu & 2,127 & Completed \\
eq1b.cpu & 1,179 & Completed \\
\midrule
\textbf{Total} & \textbf{11,947} & Progressing \\
\bottomrule
\end{tabular}
\caption{1Wat+3 equilibration vlimit warnings by stage.}
\end{table}

\subsection{Root Cause Analysis}

The instability stems from \textbf{elevated Mn(III) force constants}. Comparison with BiOx+2:

\begin{table}[H]
\centering
\begin{tabular}{lrrl}
\toprule
\textbf{System} & \textbf{Mean k (kcal/mol$\cdot$\AA$^2$)} & \textbf{vlimit} & \textbf{Status} \\
\midrule
BiOx+2 (Mn(II)) & 29.7 & 0 & Stable \\
1Wat+3 (Mn(III)) & 85--125 & 11,947 & Unstable \\
\bottomrule
\end{tabular}
\caption{Force constant comparison between stable and unstable systems.}
\end{table}

The Mn(III) oxidation state exhibits Jahn-Teller distortion, leading to anisotropic coordination geometry that MCPB.py captures as high, anharmonic force constants. Classical harmonic potentials struggle to represent this behavior, causing:
\begin{itemize}
    \item Large energy corrections for small geometric deviations
    \item Velocity spikes triggering vlimit warnings
    \item Potential for SHAKE failures in severe cases
\end{itemize}

\subsection{Implications}

This finding has important implications for OxDC mechanistic studies:
\begin{enumerate}
    \item \textbf{Mn(II) systems are more tractable} for classical MD
    \item \textbf{QM/MM may be required} for studying Mn(III) catalytic intermediates
    \item \textbf{Force constant threshold} of $\sim$35 kcal/mol$\cdot$\AA$^2$ appears predictive of stability
\end{enumerate}

Production runs for 1Wat+3 are not recommended until parameterization issues are addressed.

%======================================================================
\section{Summary of All Generated Figures}
%======================================================================

\begin{table}[H]
\centering
\begin{tabular}{lll}
\toprule
\textbf{Figure} & \textbf{Content} & \textbf{Key Finding} \\
\midrule
prod\_rmsd.png & Structural stability & Active site RMSD = 1.77~\AA \\
prod\_mn1\_coordination.png & Mn1-ligand distances & 0 dissociations \\
prod\_oxalate\_binding.png & Oxalate binding & 92.6\% bidentate \\
prod\_lid\_dynamics.png & Lid position & 5VG3-like (Glu162-out) \\
prod\_rmsf.png & Flexibility & Lid stabilized \\
prod\_correlations.png & Metric correlations & Lid independent \\
\midrule
eq\_rmsd\_ca.png & Equilibration RMSD & Stable eq \\
eq\_rmsf\_ca.png & Equilibration RMSF & Normal profile \\
eq\_energy.png & Energy components & Stable energetics \\
\midrule
bond\_energy\_distribution.png & System comparison & BiOx+2 most stable \\
force\_constant\_analysis.png & k values & BiOx+2 lowest \\
oxidation\_state\_analysis.png & Mn(II) vs Mn(III) & Mn(III) problematic \\
substrate\_coordination.png & Oxalate params & Asymmetric bidentate \\
distance\_vs\_forceconstant.png & r$_0$ vs k & Flexibility = stability \\
\bottomrule
\end{tabular}
\caption{Complete figure inventory with key findings.}
\end{table}

%======================================================================
\section{Conclusions}
%======================================================================

\subsection{What We Have Established}

\begin{enumerate}
    \item \textbf{BiOx+2 is a stable, well-behaved system}
    \begin{itemize}
        \item MCPB.py parameterization produces appropriate force constants
        \item 474 ns production trajectory is well-converged ($<$3\% SEM)
        \item Can serve as reference for comparison with other systems
    \end{itemize}

    \item \textbf{Oxalate binding is persistent and asymmetric}
    \begin{itemize}
        \item Consistent with experimental ENDOR data
        \item Bidentate mode maintained 92.6\% of 474 ns simulation
        \item Asymmetry provides mechanical flexibility
        \item May sterically prevent Glu162-in pose
    \end{itemize}

    \item \textbf{The lid maintains the 5VG3-like Glu162-out state throughout 474 ns}
    \begin{itemize}
        \item Glu162-Mn = 12.0 $\pm$ 0.7~\AA\ matches 5VG3 crystal structure ($\sim$10-12~\AA)
        \item This is \textbf{not} the open-loop state (1J58, $\sim$15-16~\AA)
        \item This is \textbf{not} the catalytic Glu162-in state (1UW8, $\sim$4.6-5.1~\AA)
        \item Zero transitions toward Glu162-in observed over 474 ns
    \end{itemize}
\end{enumerate}

\subsection{What Remains Unknown}

\begin{enumerate}
    \item \textbf{Is the backbone truly ``closed'' vs ``open-loop''?}
    \begin{itemize}
        \item Need RMSD of lid residues to 1J58 vs 1UW8 reference structures
        \item Glu162-Mn distance alone reports sidechain, not backbone
    \end{itemize}

    \item \textbf{Can Glu162 adopt the catalytic ``in'' pose?}
    \begin{itemize}
        \item Even 474 ns showed zero transitions toward Glu162-in
        \item Bidentate oxalate likely prevents it entirely (steric clash)
        \item Monodentate transition or enhanced sampling may be required
    \end{itemize}

    \item \textbf{Does Mn oxidation state affect lid dynamics?}
    \begin{itemize}
        \item 1Wat+3 (Mn(III)) equilibration shows numerical instability (Section 3)
        \item Classical MD may be inadequate for Mn(III) systems
        \item QM/MM approaches may be required for oxidation state comparisons
    \end{itemize}
\end{enumerate}

%======================================================================
\section{Next Steps}
%======================================================================

\subsection{Immediate Priorities (Next 2 Weeks)}

\begin{enumerate}
    \item \textbf{Backbone conformation analysis}
    \begin{itemize}
        \item Calculate RMSD of lid backbone to 1J58 (open) vs 1UW8 (closed) references
        \item Track Glu162 C$\alpha$-Mn1 distance (backbone proxy)
        \item Verify loop is in ``closed-backbone'' state throughout 474 ns
    \end{itemize}

    \item \textbf{Extended production analysis}
    \begin{itemize}
        \item Continue trajectory beyond 474 ns if microsecond timescales desired
        \item At 467 ns/day on B200, 1 $\mu$s requires $\sim$2.1 days
        \item Evaluate whether Glu162-in is kinetically accessible
    \end{itemize}

    \item \textbf{Address 1Wat+3 parameterization}
    \begin{itemize}
        \item Evaluate alternative Mn(III) parameterization strategies
        \item Consider reduced force constants or QM/MM hybrid approach
        \item Key question: Can classical MD ever capture Mn(III) behavior?
    \end{itemize}
\end{enumerate}

\subsection{Medium-Term Goals (Spring Semester)}

\begin{enumerate}
    \item \textbf{Enhanced sampling methods}
    \begin{itemize}
        \item Metadynamics with Glu162-Mn distance as collective variable
        \item Estimate free energy barrier for Glu162-out $\rightarrow$ Glu162-in transition
    \end{itemize}

    \item \textbf{Comparative analysis}
    \begin{itemize}
        \item BiOx+2 (Mn(II), Glu162-out) vs 1Wat+3 (Mn(III), ??)
        \item Determine oxidation state effect on lid equilibrium
    \end{itemize}

    \item \textbf{Publication preparation}
    \begin{itemize}
        \item Draft manuscript on lid dynamics and substrate binding constraints
        \item Target: J. Phys. Chem. B or J. Chem. Inf. Model.
    \end{itemize}
\end{enumerate}

%======================================================================
\section{Files and Repository}
%======================================================================

All analysis files are located in:

\begin{verbatim}
oxdc-md-fall25/
|-- systems/BiOx+2/
|   |-- PRODUCTION_ANALYSIS_REPORT.md    # Full scientific report
|   |-- analysis_scripts/
|   |   |-- analyze_production.py         # Main analysis script
|   |   |-- BACKBONE_CONFORMATION_ANALYSIS.md  # Follow-up spec
|   |-- cpptraj_ins/
|   |   |-- backbone_analysis.in          # Ready-to-run cpptraj
|   |-- analysis_results/
|       |-- figures/                       # All generated plots
|       |-- *.dat                          # Raw cpptraj output
|-- presentation/
|   |-- oxdc_md_analysis_with_figures.tex  # Updated beamer slides
|-- reports/
    |-- pi_progress_report_jan2026.tex     # This document
\end{verbatim}

Git branch: \texttt{claude/oxdc-repo-prep-EjqCu}

%======================================================================
\section*{Acknowledgments}
%======================================================================

This work was supported by the University of Florida Department of Chemistry and utilized the HiPerGator computing cluster.

%======================================================================
\section*{References}
%======================================================================

\begin{enumerate}
    \item Just VJ et al. (2004) A closed conformation of \textit{B. subtilis} oxalate decarboxylase. \textit{J Biol Chem} 279:19867-75.

    \item Saylor BT et al. (2008) The identity of the active site and importance of lid conformations. \textit{Arch Biochem Biophys} 472:114-22.

    \item Zhu J et al. (2024) Bidentate Substrate Binding Mode in Oxalate Decarboxylase. \textit{Molecules} 29:4414.

    \item Zhu W et al. (2016) Substrate Binding Mode and Molecular Basis of a Specificity Switch in OxDC. \textit{Biochemistry} 55:2163-73.

    \item Li P \& Merz KM (2016) MCPB.py: A Python Based Metal Center Parameter Builder. \textit{J Chem Inf Model} 56:599-604.
\end{enumerate}

\end{document}
